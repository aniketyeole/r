\documentclass{book}
\usepackage{blindtext}
\usepackage[utf8]{inputenc}
\usepackage[english]{babel}
\usepackage[top=2.5cm,bottom=2.5cm,left=2.5cm,right=2.5cm]{geometry}
\usepackage{fancyhdr}
\pagestyle{fancy}
 \author{Aniket Yeole}
\title{LoRa and OPC-UA based secure Industry 4.0 Application Scenario}
\date{\today}


\begin{document}
\maketitle
\tableofcontents
\chapter{Introduction}
\section{Scope for LPWAN in Industry 4.0 Application}
\section{Purpose}
\section{Delimitation}

\chapter{State of Art}
\section{}

\chapter{Technical Background}
\section{Fundamentals of IoT protocol}
\subsection{LoRa}
\subsection{SigFox}
\subsection{NarrowBand IoT}
\section{LoRa vs LoRaWAN}
\subsection{Architecture}
\subsection{LoRaWAN operation modes}
\subsection{The Things Network}
\section{MQTT Protocol}
\section{OPC-UA}
\subsection{OPC-UA Architecture}
\subsection{OPC-UA Server}
\subsection{OPC-UA Client}
\section{ProfiBus}
\subsection{background}

\chapter{Security Aspect for LoRaWAN network}
\section{Current Security Features and Vulnerabilities}
\subsection{AES Encryption}
\subsection{Data Encryption}
\subsection{Identifiers}
\subsection{Multicast vs Unicast}
\section{Security at Architectural Level}
\subsection{End Devices}
\subsection{Gateway}
\subsection{Network Server}
\subsection{Application Server}

\chapter{System Design}
\section{LoRa as LPWAN choice}
\subsection{Network Server}
\section{OPC-UA as a choice}
\section{MQTT for bridging}
\section{ProfiNet}

\chapter{Implementation and Results}
\section{Inter operable LoRa and OPC-UA}
\subsection{Bridging LoRaWAN and OPC-UA}
\section{LoRa and Profinet}
\section{Results}
\subsection{*}

\chapter{Conclusion}
\section{Future Work}

\chapter{Appendix}

\end{document}